\documentclass[11pt]{article}
\usepackage{a4wide}
\usepackage{amsmath}
\usepackage{amssymb}
\usepackage{amsthm}
\usepackage{cite}

\newtheorem{problem}{Problem}[section]
\newtheorem{definition}{Definition}[section]
\newtheorem{theorem}{Theorem}[section]
\newtheorem{lemma}[theorem]{Lemma}
\newtheorem{proposition}[theorem]{Proposition}
\newtheorem{corollary}[theorem]{Corollary}

\author{M.~El-Kebir, G.W.~Klau}
\title{Fragment-based molecule parameterisation}

\begin{document}

\maketitle

\section{Introduction}

The automated topology builder (ATB) is a web server for generating topologies
of novel molecules compatible with the GROMOS 53A6 force field \cite{Malde11}.
The ATB is able to parameterise molecules consisting of up to 50 atoms. Due to
the complexity of the quantum mechanics computations, molecules larger than 50
atoms pose a problem for the ATB. 

Here, we introduce a method that assists the ATB by searching for fragments
common to both the input molecule and a molecule in the repository. The atom
charges of the input molecule are set according to the charges of the core atoms
of the found common fragments.

\section{Problem definition}

A molecule is a simple graph $G=(V,E)$ whose nodes and edges correspond to atoms
and bonds, respectively. Nodes are labeled by their partial charge $w : V
\rightarrow \mathbb{R}$ and their atom type $t : V \rightarrow \mathbb{N}$. 
For a subset $V^\prime \subseteq V$, $t(V^\prime)$ is the set $\bigcup_{v \in
V^\prime} t(v)$. The neighborhood $N(v)$ of a node $v \in V$ is the set $\{ u
\mid (u,v) \in E \}$. Similarly for a subset $V^\prime \subseteq V$, we define
$N(V^\prime)$ to be the set $\bigcup_{v \in V^\prime} N(v)$. 

Given molecules $G_1 = (V_1, E_1)$ and $G_2 = (V_2, E_2)$ with atom types $t_1 :
V_1 \rightarrow \mathbb{N}$ and $t_2 : V_2 \rightarrow \mathbb{N}$, we define a
common fragment and its shell as follows.

\begin{definition}
  A \emph{common fragment} is a pair $(V^\prime_1, V^\prime_2)$ with $V^\prime_1
  \subseteq V_1$ and $V^\prime_2 \subseteq V_2$ that admits a bijection $h :
  V^\prime_1 \cup N(V^\prime_1) \rightarrow V^\prime \cup N(V^\prime_2)$ such
  that
  \begin{enumerate}
    \item[(i)] $G_1[V^\prime_1]$ and $G_2[V^\prime_2]$ are connected,
    \item[(ii)] $(u,v)$ is an edge in $G_1[V^\prime_1 \cup N(V^\prime_1)]$ if and
      only if $(h(u),h(v))$ is an edge in $G_2[V^\prime_2 \cup N(V^\prime_2)]$,
    \item[(iii)] $t_1(v) = t_2(h(v))$ for all $v \in V^\prime_1 \cup
      N(V^\prime_1)$.
  \end{enumerate}
\end{definition}
\begin{definition}
  The \emph{shell} of a common fragment $(V^\prime_1, V^\prime_2)$ is given
  by $(N(V^\prime_1), N(V^\prime_2))$.
\end{definition}

\begin{definition}
A common fragment $(V^\prime_1,V^\prime_2)$ is \emph{maximal} if
there exists no common fragment $(V^{\prime\prime}_1,V^{\prime\prime}_2)$ such
that $V^\prime_1 \subseteq V^{\prime\prime}_1$ and $V^\prime_2 \subseteq
V^{\prime\prime}_2$.
\end{definition}

The problem that we want to solve is now as follows.

\begin{problem}
Given graphs $G_1 = (V_1, E_1)$ and $G_2 = (V_2, E_2)$ with atom types $t_1 :
V_1 \rightarrow \mathbb{N}$ and $t_2 : V_2 \rightarrow \mathbb{N}$, find the
set of all maximal common fragments.
\end{problem}

\section{Method}

We solve the problem by finding maximal cliques on the node product graph, which
is defined as follows.

\begin{definition}
The \emph{node product graph} $G_1 \otimes G_2$ has a node set $\{(u,v) \in V_1
\times V_2 \mid t_1(u) = t_2(v) \mbox{ and } t_1(N(u)) = t_2(N(v)) \}$ and an
edge between $(u,v)$ and $(u^\prime,v^\prime)$ if and only if $(u,u^\prime) \in
E_1$ and $(v,v^\prime) \in E_2$, or $(u,u^\prime) \not \in E_1$ and
$(v,v^\prime) \not \in E_2$.
\end{definition}

\begin{itemize}
  \item Bron-Kerbosch algorithm
  \begin{itemize}
    \item Adjusted to find connected common subgraphs~\cite{Koch:1996fc,Koch:2001wi}
  \end{itemize}
\end{itemize}

\bibliographystyle{abbrv}
\bibliography{fragments}

\end{document}

